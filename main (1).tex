\documentclass{article}
\usepackage{graphicx} % Para adicionar imagens
\usepackage{float} % Para fixar a posição das imagens

\title{Álgebra Linear - PCA}
\author{Fernanda Rafaela}
\date{Novembro, 2024}

\begin{document}

\maketitle

\section{Dataset}
O dataset inclui fatores de risco que auxiliam na previsão e prevenção de doenças cardiovasculares, possibilitando a análise de elementos que influenciam a mortalidade por insuficiência cardíaca.\\
Link: \url{https://www.kaggle.com/datasets/andrewmvd/heart-failure-clinical-data}

\section{Desenvolvimento}
Para realizar a normalização dos dados e aplicar o algoritmo de PCA, serão usadas as seguintes bibliotecas no Python:

\begin{figure}[H]
    \centering
    \includegraphics[width=0.8\textwidth]{biblioteca.png}
    \caption{Bibliotecas utilizadas para análise}
    \label{fig:bibliotecas}
\end{figure}

Primeiro, importamos o arquivo CSV e normalizamos o dataset, calculando a média de cada coluna e subtraindo do total:

\begin{figure}[H]
    \centering
    \includegraphics[width=0.8\textwidth]{csv.png} 
    \caption{Importação do CSV}
    \label{fig:import_csv}
\end{figure}

Em seguida, calculamos a matriz de covariância, transpondo-a:

\begin{figure}[H]
    \centering
    \includegraphics[width=0.8\textwidth]{cov.png}
    \caption{Matriz de Covariância}
    \label{fig:covariancia}
\end{figure}

Agora, calculamos os autovetores e autovalores e ordenamos do maior para o menor:

\begin{figure}[H]
    \centering
    \includegraphics[width=0.8\textwidth]{autos.png}
    \caption{Cálculo dos Autovalores e Autovetores}
    \label{fig:autovalores}
\end{figure}

Maiores autovalores e autovetores:

\begin{figure}[H]
    \centering
    \includegraphics[width=0.8\textwidth]{maiores.png}
    \caption{Exibição dos maiores autovetores e autovalores}
    \label{fig:autovalores}
\end{figure}

Usando os 2 maiores autovalores calculamos o pca e fazemos um grafico bidimensional:

\begin{figure}[H]
    \centering
    \includegraphics[width=0.8\textwidth]{2dcode.png}
    \caption{Criação mapa 2d}
    \label{fig:autovalores}
\end{figure}

\begin{figure}[H]
    \centering
    \includegraphics[width=0.8\textwidth]{2d.png}
    \caption{Mapa 2d}
    \label{fig:autovalores}
\end{figure}

Usando os maiores 3 autovetores calculamos o pca e fazemos o gráfico tridimensional: 

\begin{figure}[H]
    \centering
    \includegraphics[width=0.8\textwidth]{3dCode.png}
    \caption{Criação mapa 3d}
    \label{fig:autovalores}
\end{figure}

\begin{figure}[H]
    \centering
    \includegraphics[width=0.8\textwidth]{3d.png}
    \caption{Mapa 3d}
    \label{fig:autovalores}
\end{figure}

\end{document}
